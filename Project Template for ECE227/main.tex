\documentclass[journal]{IEEEtran}
% \documentclass[12pt,letterpaper]{article}
\usepackage[margin=1in]{geometry}
\usepackage[english]{babel}
\usepackage[utf8x]{inputenc}
\usepackage{amsmath}
\usepackage{amssymb} 
% \usepackage[retainorgcmds]{IEEEtrantools}
\usepackage{graphicx}
\usepackage{tabularx}
\usepackage{kpfonts}    % for nice fonts
\usepackage{microtype} 
\usepackage{booktabs}   % for nice tables
\usepackage{bm}         % for bold math
\usepackage{listings}   % for inserting code
\usepackage{verbatim}   % useful for program listings
\usepackage{color}
\usepackage{subcaption}
\usepackage{booktabs}
\usepackage{array}
\usepackage[colorlinks=true]{hyperref}
\usepackage{pdflscape}
% use for hypertext
% \usepackage[colorinlistoftodos]{todonotes}
\usepackage{natbib}

\usepackage{xcolor}
\usepackage{subcaption}

\newtheorem{definition}{Definition}
\newtheorem{theorem}{Theorem}
\newtheorem{proposition}{Proposition}
\newtheorem{corollary}{Corollary}
\newtheorem{lemma}{Lemma}
\newtheorem{example}{Example}

\usepackage{lipsum}

\begin{document}

\title{Exploring Impact of Trust in Prisoner's Dilemma on Social Networks\\ \Large ECE 227}
        
    \author{
    \IEEEauthorblockN{Karl Hernandez},
    \IEEEauthorblockA{\textit{Department of Electrical and Computer Engineering} \\
    \textit{UC San Diego}\\
    kphernan@ucsd.edu}

    \and
    
    \IEEEauthorblockN{Achyut Pillai},
    \IEEEauthorblockA{\textit{Department of Electrical and Computer Engineering} \\
    \textit{UC San Diego}\\
    apillai@ucsd.edu}
    \and
    
    \IEEEauthorblockN{Alexis Yu},
    \IEEEauthorblockA{\textit{Department of Electrical and Computer Engineering} \\
    \textit{UC San Diego}\\
    ahyu@ucsd.edu}
    }
    
\maketitle
%-Title
%+Abstract
\begin{abstract}
In this paper, we investigate the dynamics of cooperation in the Prisoner's Dilemma (PD) when played on two real world social networks, the undirected Facebook "circles" network and the directed Epinions network with trust signing.
For each network, we implement two classical payoff-based update rules, \emph{imitate best neighbor} and \emph{Fermi}, to observe how cooperation evolves under standard PD dynamics.
The fraction of cooperators and the number of iterations to convergence over three bernoulli initial distributions (\(p \in \{0.25, 0.50, 0.75\}\)) are recorded.
We then introduce two trust aware update rules on the Epinions network, one where agents only consider trusted neighbors when updating strategy and another where agents aggregate trust from all neighbors.
Our results show that in the dense Facebook network, the standard update rules converge rapidly and to moderate levels only when there is high initial cooperation.
The sparse Epinions network, however yields low cooperation and converges slightly slower.
The pairwise trust boosts cooperation significantly, across all initial conditions, while the all-neighbor trust rule drives cooperation above 90\% and converges very quickly.
These results demonstrate that embedding trust into the update rules can drastically enhance both the level and speed of convergence to cooperation, especially in more sparse networks.
\end{abstract}
%-Abstract

\captionsetup{font=footnotesize} % or small


\section{Introduction}
The Prisoner’s Dilemma is a two‐player game in which each participant can either cooperate or defect. The outcomes are either mutual cooperation, yielding a moderate reward for both; unilateral defection, giving the defector a high payoff and the cooperator a low payoff; and mutual defection, leaving both with a low payoff. 

In this paper, we examine the Prisoner's Dilemma’s dynamics on two empirically derived social networks: the Facebook Social Circles (from SNAP) and the Epinions graph (also from SNAP). On each network, we first employ a standard "imitate the best" update rule, whereby each node plays PD with all neighbors, counts payoffs, and then takes the highest payoff neighbor's strategy. Then, on the Epinions network, we suggest a trust‐sensitive version: after each round, agents only consider neighbors that they trust when determining whom to copy.

\section{Related Works}
A substantial body of past research has explored how network structure and trust relationships influence cooperation in an iterative Prisoner’s Dilemma setting. Cameron and Cintrón-Arias revisit the PD on real social networks, demonstrating that empirical topologies, characterized by heterogeneous degree distributions and community structure, play a crucial role in the spread and stability of cooperative strategies. High-degree nodes and clustering sometimes facilitate and sometimes obstruct cooperation, depending on the payoffs and update rules. Dynamic social networks facilitate cooperation in the N-player Prisoner’s Dilemma by Rezaei and Kirley and they build on this by introducing dynamic rewiring in the N-player PD. This allows agents to adjust ties based on past payoffs so that defectors lose links and cooperators attract new ones, which leads to defectors being isolated while cooperators cluster together. Meanwhile, Trust-induced cooperation under the complex interaction of networks and emotions by Xie et al. looks at “trust-induced cooperation” under complex network and emotional interactions by embedding a continuous, weighted trust metric into the agent’s decisions. This stabilized cooperation under harsher payoff conditions and produces more accurate patterns of cooperation compared to real social behavior.

         
\section{Model}
\begin{figure*}[ht]
    \centering
    \includegraphics[width=0.3\linewidth]{figures/Calls_network.png}
    \includegraphics[width=0.3\linewidth]{figures/FB_network.png}
    \includegraphics[width=0.3\linewidth]{figures/ff.png}
    \caption{Figure example}
    \label{fig:enter-label}
\end{figure*}


The first network we looked at was the Facebook dataset (SNAP). This network consists of anonymized ego-networks extracted from Facebook, where each node represents a user and edges represent friendship links. Using this as our base, we assigned each node’s initial strategy with a Bernoulli (0.5) distribution, giving an equal chance between cooperation and defection. Then, after every game iteration playing Prisoner’s Dilemma, there is an update strategy where each node will look at all its neighbors and will adopt the more successful strategy. 
The next network we looked at was the Epinions social network dataset (SNAP), a directed “who-trusts-whom” graph from the Epinions.com review site, where each edge indicates that one user has marked another as trusted or not trusted. 

\section{Dataset Description}
\lipsum[7-10]
\begin{figure}[ht]
    \centering
    \includegraphics[width=0.8\linewidth]{figures/plot_fraction_vs_n_for_single_multiplex_network.png}
    \caption{Figure example}
    \label{fig:enter-label}
\end{figure}

\section{Simulations and Numerical Experiments}
\begin{figure*}[ht]
    \centering
    \includegraphics[width=0.4\linewidth]{figures/tw.png}
    \includegraphics[width=0.4\linewidth]{figures/SMS_network.png}
    \caption{Figure example}
    \label{fig:enter-label}
\end{figure*}
\lipsum[9-12]
\section{Conclusion}
Our study confirms that there is a connection between network structure and the update rules used in the prisoner's dilemma, that determines the dynamics of cooperation. 
While the classical \emph{imitate-best-neighbor} and \emph{Fermi} update rules are effective enough in encouraging cooperation in highly clustered networks like Facebook, under favorable initial conditions,
they are not sufficient to maintain cooperation in sparse networks like Epinions. By contrast, the \emph{trust-aware} update rules, which incorporate trust relationships, 
substantially enhance cooperation levels and convergence speed in the Epinions network.
Most notably, the \emph{all-neighbor-trust} update rule achieves over 90\% cooperation across all initial conditions with relatively fast convergence, thus demonstrating the power of trust in fostering cooperation.
These findings suggest that real social networks, may require leveraging trust relationships to maintain cooperation as well as to overcome obstacles presented by sparse network structures.

\section{Future works}
Future work could proceed in several directions to further explore the dynamics of cooperation in the prisoner's dilemma on social networks.
One potential direction is to introduce adaptive rewiring mechanisms where agents can change their connections and trust based on past interactions similar to the work by Rezaei and Kirley \cite{RezaeiKirley2012}.
Next, we could investigate the impact of different trust models, such as those with weighted trust or trust decay based on length of connection and interaction history, to see how the dynamics of trust affect the dynamics of cooperation.
Another direction is to see how a varying payoff matrix affects the dynamics, such as with different reward structures for cooperation and defection or a payoff dependent on a cost function that changes over time.
Finally, we could explore the experiments on more real social networks, such as Twitter, Reddit, or GitHub, to see how different real-world structures affect the dynamics.

\clearpage
\bibliographystyle{apalike}
\bibliography{aguiar}

\newpage

\begin{IEEEbiography}[{\includegraphics[width=1in,height=1.25in,clip,keepaspectratio]{figures/karl.jpg}}]{Karl Hernandez} earned his B.S. degree in Computer Engineering at the University of California, San Diego in 2024, and is now pursuing his M.S. degree in Electrical and Computer Engineering with a specialization in Machine Learning and Data Science at the same institution. His research interests include privacy-preserving machine learning, and federated agentic AI.
\end{IEEEbiography}
\vspace{-1in}
\begin{IEEEbiography}[{\includegraphics[width=1in,height=1.25in,clip,keepaspectratio]{figures/achyut.jpg}}] {Achyut Pillai} earned his B.S. degree in Electrical Engineering from the University of California, San Diego in 2024. He is currently pursuing his M.S. degree in Electrical and Computer Engineering with a focus on Machine Learning and Data Science at the University of California, San Diego. His research interests include machine learning, network science, and large language models.
\end{IEEEbiography}
\vspace{-1in}
\begin{IEEEbiography}[{\includegraphics[width=1in,height=1.25in,clip,keepaspectratio]{Project Template for ECE227/figures/Alexis (2).JPG}}]{Alexis Yu} earned her B.S. degree from University of California, San Diego in Electrical Engineering with a depth in Computer System Design in 2024. She is currently pursuing her Master's degree in Computer Engineering at the University of California, San Diego. She will be pursuing ASIC and FPGA design in the future. 
\end{IEEEbiography}
\end{IEEEbiography}

\end{document}
